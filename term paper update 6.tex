\documentclass{article}
\begin{document}
\title{Gold Nanoparticles in Nuclear Medicine }
\author{Alan Biju}


\maketitle


\section*{Abstract}

gold nanoparticles have gained increased interest as radiosensitizing, photothermal therapy and optical imaging agents to enhance the effectiveness of cancer detection and therapy.in this paper we will be focusing about nuclear medicine in general and how gold nanoparticles will be useful. unique optical and physicochemical properties, gold nanoparticles have gained increased interest as in radiosensitizing, photothermal therapy, and optical imaging agents to enhance the effectiveness of cancer detection and therapy. Furthermore, their ability to carry multiple medically relevant radionuclides broadens their use to nuclear medicine SPECT and PET imaging as well as targeted radionuclide therapy. In this review, we discuss the radiolabeling process of gold nanoparticles and their use in (multimodal) nuclear medicine imaging to better understand their specific distribution, uptake, and retention in different in vivo cancer models. In addition,
radiolabeled gold nanoparticles enable image-guided therapy is reviewed as well as the enhancement of targeted radionuclide therapy and nanobrachytherapy through an increased dose deposition and radiosensitization, as demonstrated by multiple Monte Carlo studies and experimental in vitro and in vivo studies.


\section*{ An introduction in nuclear medicine}
Nuclear medicine involves the internal administration of radionuclides to diagnose, stage, treat, and follow-up of diseases, including cancer. Radiopharmaceuticals are developed by linking a radionuclide to a carrier molecule (also referred to as the targeting molecule), which is directed against a cancer-specific antigen or process. The selection of the suitable radionuclide depends on its specific emission and the intended application. In more detail, positron (β+ particles)- and gamma-emitting radionuclides enable 3D positron emission tomography (PET)and single-photon emission computed tomography (SPECT) imaging, respectively. Consequently, the radiopharmaceutical can be traced inside the body providing functional information about specific molecular and cellular processes in the tumor depending on the carrier molecule, such as blood flow, metabolism, receptor expression, tumor metastatic capacity, inflammation, programmed cell death. On the other hand, radionuclides emitting β− particles (e.g. iodine-131, lutetium-177,yttrium-90), α-particles (e.g. actinium-225, astatine-221, bismuth-213, lead-212) or Auger electrons (e.g. iodine-125, iodine-123,indium-111, terbium-161, gallium-67), which are coupled to a cancer-targeting molecule, have the potential to deliver a cytotoxic radiation dose to the cancer cells. This therapeutic strategy is called targeted radionuclide therapy (TRT). TRT is a rapidly growing field. Some recent examples are the development of radiolabeled prostate-specific membrane antigen (PSMA) and the approval of [177Lu]Lu-DOTA-TATE to treat neuroendocrine tumors. However, research continues to investigate how to maximize the benefit of radionuclide therapies that are effective and safe for each individual patient 