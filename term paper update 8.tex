\documentclass{article}
\begin{document}
\title{Gold Nanoparticles in Nuclear Medicine }
\author{Alan Biju}


\maketitle


\section*{Abstract}

gold nanoparticles have gained increased interest as radiosensitizing, photothermal therapy and optical imaging agents to enhance the effectiveness of cancer detection and therapy.in this paper we will be focusing about nuclear medicine in general and how gold nanoparticles will be useful. unique optical and physicochemical properties, gold nanoparticles have gained increased interest as in radiosensitizing, photothermal therapy, and optical imaging agents to enhance the effectiveness of cancer detection and therapy. Furthermore, their ability to carry multiple medically relevant radionuclides broadens their use to nuclear medicine SPECT and PET imaging as well as targeted radionuclide therapy. In this review, we discuss the radiolabeling process of gold nanoparticles and their use in (multimodal) nuclear medicine imaging to better understand their specific distribution, uptake, and retention in different in vivo cancer models. In addition,
radiolabeled gold nanoparticles enable image-guided therapy is reviewed as well as the enhancement of targeted radionuclide therapy and nanobrachytherapy through an increased dose deposition and radiosensitization, as demonstrated by multiple Monte Carlo studies and experimental in vitro and in vivo studies.


\section*{ An introduction in nuclear medicine}
Nuclear medicine involves the internal administration of radionuclides to diagnose, stage, treat, and follow-up of diseases, including cancer. Radiopharmaceuticals are developed by linking a radionuclide to a carrier molecule (also referred to as the targeting molecule), which is directed against a cancer-specific antigen or process. The selection of the suitable radionuclide depends on its specific emission and the intended application. In more detail, positron (β+ particles)- and gamma-emitting radionuclides enable 3D positron emission tomography (PET)and single-photon emission computed tomography (SPECT) imaging, respectively. Consequently, the radiopharmaceutical can be traced inside the body providing functional information about specific molecular and cellular processes in the tumor depending on the carrier molecule, such as blood flow, metabolism, receptor expression, tumor metastatic capacity, inflammation, programmed cell death. On the other hand, radionuclides emitting β− particles (e.g. iodine-131, lutetium-177,yttrium-90), α-particles (e.g. actinium-225, astatine-221, bismuth-213, lead-212) or Auger electrons (e.g. iodine-125, iodine-123,indium-111, terbium-161, gallium-67), which are coupled to a cancer-targeting molecule, have the potential to deliver a cytotoxic radiation dose to the cancer cells. This therapeutic strategy is called targeted radionuclide therapy (TRT). TRT is a rapidly growing field. Some recent examples are the development of radiolabeled prostate-specific membrane antigen (PSMA) and the approval of [177Lu]Lu-DOTA-TATE to treat neuroendocrine tumors. However, research continues to investigate how to maximize the benefit of radionuclide therapies that are effective and safe for each individual patient 

\section*{The potential advantages of nanoparticles in nuclear medicine}
A ‘nanomaterial’ is defined as a natural, incidental, or manufactured material with one or more external dimensions in the size range of 1 nm to 100 nm. In this size range, material properties become controllable [6]. Hence, nanoparticles can arise in several shapes, such as spheres, rods, discs, cubes, and cages. Furthermore, as the size of the nanoparticles decreases, their surface-area-to-volume ratio is strongly increasing. Thanks to these specific properties, nanoparticles can offer a significant contribution to nuclear medicine. First, a major advantage is the potential of a single nanoparticle to hold multiple radionuclides, achieving much higher payloads of radioactivity as compared to a conventional radiopharmaceutical agent that carries only one or a few radionuclides (Fig. 1A). In fact, Lucas, et al. calculated in a Monte Carlo simulation that nanoparticles containing multiple β—emitters (yttrium-90, lutetium-177, iodine-131, iodine-124 or rhenium-188) may deliver a total absorbed radiation dose of >60 Gyto a solid, non-small-cell lung carcinoma model, which could not be achieved by antibodies that were each conjugated to a single radionuclide [7]. Second, the predominant theory is that due to their small size, nanoparticles can efficiently extravasate through the gaps between endothelial cells of the leaky and immature blood vessels into the tumor mass. Furthermore, the decreased level of lymphatic drainage of the interstitial fluid within the tumor contributes to nanoparticle tumor retention. This rationale is known as the enhanced permeability and retention(EPR) effect and causes the accumulation and prolonged retention of radiolabeled nanoparticles in the tumor tissue, increasing the tumor radiation dose [8,9]. However, it is important to point out that despite the effect is tremendously successful in preclinical animal models, the clinical efficacy and translation of cancer nanomedicines remains poor, indicating that the EPR phenomenon is less reliable in human cancers [10–13]. Therefore, interest is growing in the extravasation of nanoparticles into tumors via active transendothelial pathways, which appears not to be
underestimated. In fact, Sindhwani, et al. demonstrated a very low frequency of interendothelial gaps in different xenograft models, such asU87-MG glioblastoma, 4T1 breast cancer, genetically engineered MMTVPyMT breast cancer, and patient-derived breast cancer, as well as in biopsies of human ovarian, breast and glioblastoma tumors. In contrast, fenestrae and vacuoles, which are associated with endothelial transcytosis, occur much more frequently in the tumor vasculature across all models. Furthermore, by deactivating active transendothelial transport pathways, the authors concluded that only 3–25%  of nanoparticle tumor entry isattributed to the passive transport through gaps, depending on the nanoparticle size 